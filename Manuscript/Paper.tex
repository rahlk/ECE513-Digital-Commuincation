\documentclass[10pt,journal,compsoc,onecolumn
]{IEEEtran} %draftclsnofoot, 
\usepackage{graphicx}
\graphicspath{ {./_figures/} }
\setlength\fboxsep{1pt}
\setlength\fboxrule{1pt}
\usepackage{multicol}
\bstctlcite{IEEEexample:BSTcontrol}
\usepackage{bigstrut}
\usepackage{tabularx}
\usepackage{tabulary}
\usepackage{booktabs}
\usepackage{amsmath}
\usepackage{balance}
\usepackage{flushend}
\setlength{\parindent}{0em}
\setlength{\parskip}{1em}
\usepackage[caption=false,font=normalsize,labelfont=sf,textfont=sf]{subfig}
\usepackage{hyperref}
\hypersetup{
  colorlinks = false,
  hidelinks = true
	}
\usepackage[english]{babel}
\usepackage{blindtext}
\usepackage{times}
\usepackage{cite}

\begin{document}
  \markboth{BME 560 Medical Imaging: X-ray, CT, and Nuclear Methods. Fall, 
  2014}%
  {BME 560 Medical Imaging: X-ray, CT, and Nuclear Methods. Fall, 2014}
  
  \title{Data Detection for Band Limited Signals with ISI}
  \author{\IEEEauthorblockN{Rahul Krishna}, 
  \IEEEauthorblockA{\normalsize {\textit{Dept. of Electrical and Computer 
  Engineering}\\
    North Carolina State University, Email: 
    \href{mailto:rkrish11@ncsu.edu}{{rkrish11@ncsu.edu}}}}\\\\
    \and 
    \IEEEauthorblockN{Hongwei Wang}, 
  \IEEEauthorblockA{\normalsize {\textit{Dept. of Electrical and Computer 
  Engineering}\\
    North Carolina State University, Email: 
    \href{mailto:hwang32@ncsu.edu}{hwang32@ncsu.edu}}}}}

	\IEEEcompsoctitleabstractindextext{%
  \begin{abstract}
This paper explores the various tools for detecting the information symbols at the receiver when the received signal is transmitted through a non-ideal band limited channel with additive Gaussian noise. We plan on studying the application of optimum Maximum Likelihood Sequence estimation (MLSE) using the Viterbi algorithm. We would also like to look into its performance for channels with intersymbol interference.

It has been shown that the MLSE for a channel with ISI has a computational complexity that grows exponentially with the length of the channel time dispersion. Therefore, other suboptimal techniques such as linear equalization (LE), and decision-feedback equalization (DFE) will be studied as alternative approaches to MLSE \cite{Proakis}. As an extension to the above, we also plan on inspecting the Iterative Correction of Intersymbol-Interference using Turbo equalization \cite{Catherine}.

Our paper will therefore be a detailed survey of the aforementioned techniques for detection data sequence for band limited signals with intersymbol interference. The primary focus however will be on maximum likelihood sequence estimation using the viterbi algorithm. After remarking on it's merits and demerits, other alternative approaches will be addressed and will be compared one another.
    
  \end{abstract}
  % IEEEtran.cls defaults to using nonbold math in the Abstract.
  % This preserves the distinction between vectors and scalars. However,
  % if the journal you are submitting to favors bold math in the abstract,
  % then you can use LaTeX's standard command \boldmath at the very start
  % of the abstract to achieve this. Many IEEE journals frown on math
  % in the abstract anyway. In particular, the Computer Society does
  % not want either math or citations to appear in the abstract.
  
  % Note that keywords are not normally used for peerreview papers.
  \begin{IEEEkeywords}
    Tomotherapy, Intensity Modulated Radiation Therapy.
  \end{IEEEkeywords}}
  \maketitle
  
  \section{Introduction}
\end{document}